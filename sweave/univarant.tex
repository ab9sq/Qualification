\documentclass[10pt]{article}

\usepackage{url}

\usepackage{hyperref}
\hypersetup{
    colorlinks,
    citecolor=black,
    filecolor=black,
    linkcolor=black,
    urlcolor=black
}

\usepackage[toc]{multitoc}
\renewcommand*{\multicolumntoc}{2}
%\setlength{\columnseprule}{0.5pt}


\title{Qualification of R Univariate}
\author{Nick Lauerman}

\usepackage{Sweave}
\begin{document}
\input{univarant-concordance}
\maketitle

\begin{abstract}
The purpose of this suite of test is to compare the output of R 32 bit build
version 3.1.2 with standard univariate test results provided by the National
Inistute of Standards and Testing (NIST)
\end{abstract}

\tableofcontents

\section{Setup}
\subsection{R}
This seriess of tests is being run on R with only the ``base'' packages or libraries
installed. The following commands are issued prior to runing hte tests for the 
reason stated

\begin{description}
   \item[options(digits = 15)] This is used to specify that 15 digits are to be displayed
   on numbers.
   \item[path] Sets the path to the directory were the data sets are stored.
\end{description}
\begin{Schunk}
\begin{Sinput}
> options(digits = 22)
> path="~/R/workspace/Qualification/raw data/"
\end{Sinput}
\end{Schunk}
\subsection{Computer Information}

The degree of accuracy  that can be expected from a computer if a function of several
factors including the processor used. R provides a method to determine 
numeric tolerance based on David Goldberg (1991), 
``What Every Computer Scientist Should Know About Floating-Point Arithmetic'', 
ACM Computing Surveys, 23/1, 5-48, also available via 
\url{http://www.validlab.com/goldberg/paper.pdf}.

This value can be treated as the error value for the computer and for accuracy 
beyond requires careful consideration.
\begin{Schunk}
\begin{Sinput}
> .Machine$double.eps ^ 0.5
\end{Sinput}
\begin{Soutput}
[1] 1.4901161193847656e-08
\end{Soutput}
\end{Schunk}
\subsection{R Information}
\begin{Schunk}
\begin{Sinput}
> version
\end{Sinput}
\begin{Soutput}
               _                           
platform       x86_64-w64-mingw32          
arch           x86_64                      
os             mingw32                     
system         x86_64, mingw32             
status                                     
major          3                           
minor          5.1                         
year           2018                        
month          07                          
day            02                          
svn rev        74947                       
language       R                           
version.string R version 3.5.1 (2018-07-02)
nickname       Feather Spray               
\end{Soutput}
\end{Schunk}

\subsection{Data Cleaning}
all data sets downloaded from NIST as a DAT (ASCII Format) file were cleaned up 
to remove header information that was imbeded in the file. The file was than saved
as a TXT file without any additional changes. This clean up was done to simplify 
the loading of the data into R.

\section{Univariate Summary Background Information}
\begin{quote}


Error Types: 

We provide datasets with certified values for the mean, standard deviation, and (lag-1) autocorrelation coefficient to assess the accuracy of Univariate Summary Statistic calculations in statistical software. Computational inaccuracy has 3 sources:
\begin{itemize}
   \item truncation error;
   \item cancellation error;
   \item accumulation error.
\end{itemize}


Truncation error is the inexact binary representation error in storing decimal numbers according to the IEEE standard arithmetic. Of course, once these representational digits are lost, they cannot be recovered; their effect can at best be held constant, and at worst propagated to larger errors. 

Cancellation error is an error that occurs when analyzing data that has low relative variation; that is, data with a high level of "stiffness". In "Assessing the Accuracy of ANOVA Calculations in Statistical Software" (Computational Statistics \& Data Analysis 8 (1989), pp 325-332) Simon and Lesage noted that as the number of constant leading digits in a particular dataset increases and the data grows more nearly constant (i.e., the stiffness increases) accurate computation of standard deviations becomes increasingly difficult. This also holds for other similarly computed summary statistics, like the autocorrelation coefficient. In both cases computation is hindered by subtracting data from a mean quite close to the data, leaving behind the digits from the mantissa of each data element that are most likely to have been misrepresented. 

Accumulation error (also as noted by Simon \& Lesage) is the error that occurs in direct proportion to the total number of arithmetic computations, which in turn in this univariate case is proportional to the number of observations. This increases the accumulation of small errors, making accurate computations difficult. 

Levels of Difficulty: 

We include both generated and "real world" datasets so as to allow computational accuracy to be examined at different stiffness levels and different accumulation error levels. We have, in a fashion similar to the ANOVA datasets, drawn from the benchmark work of Simon and Lesage (1989), and have 4 "generated" data sets with the number of constant leading digits set to 7, 1, 7, and 8, respectively, and with the number of observations set to 3, 1001, 1001, and 1001, respectively. 5 "real world" datasets were borrowed from the dataset repository of the Dataplot Statistics/Graphics software system; two of these are from NIST statistical consulting, and the other 3 are "classic" general-interest sets drawn from outside NIST. 

Datasets are ordered by level of difficulty (lower, average, and higher) according to their stiffness--the number of constant leading digits. This ordering is simply meant to provide rough guidance for the user; producing correct results on a dataset of higher difficulty does not imply that your software will correctly solve all datasets of average or even lower difficulty. Of the 9 datasets, 6 (5 "real world" and 1 generated) datasets are of the lower level of difficulty, 2 (generated) are of average level of difficulty, and 1 (generated) is of higher level of difficulty. 

Simple Remedial Action:

In computing general summary statistics, if you find your software giving less-than-desirable results in the calculation of the sample standard deviation, one simple remedial measure is to subtract the leading constant from all the observations in that dataset before analyzing it, and a second remedial measure is to assure yourself that the sample standard deviation is computed by the formula which first computes deviations about the mean before squaring and summing, as opposed to using the old desk calculator formula of a generation ago which involves the (computationally unstable) difference of 2 large numbers: the sums of squares of the raw data (uncentered) and the sum of the squared sample mean. 

As noted in the  General Background Information producing correct results for all datasets in this collection does not imply that your software will do the same for your own particular dataset. It will, however, provide some degree of assurance, in the sense that your package provides correct results for datasets known to yield incorrect results for some software
\end{quote}


\section{NumAcc4 Data Set}
Using the NumAcc4.dat \url{http://www.itl.nist.gov/div898/strd/univ/numacc4.html} file. 

\subsection*{Data Set Description}

This generated/fabricated dataset consists of 1001 9-digit floating-point values: a single 10000000.2, followed by 500 pairings of 10000000.1 and 10000000.3. By construction, this data set has sample mean = 10000000.2 (exact); sample standard deviation = .1 (exact); and sample autocorrelation coef. = -0.999 (exact). The construction was carried out based on considerations described by Simon, Stephen D. and Lesage, James P. (1989): Assessing the Accuracy of ANOVA Caluclations in Statistical Software", Computational Statistics \& data Analysis, 8, pp. 325-332. 

\begin{description}
\item[] Data set properties
\item[Level of Difficulty] Higher
\item[Variables] 1
\item[Observations] 101
\item[First Observation] 10000000.2
\end{description}

\begin{description}
\item[] Expected Results (as certified)
\item[Mean] 10000000.2 (exact)
\item[Standard Deviation]0.1 (exact)
\item[population lag-1 autocorrelation coefficient]-0.999       (exact)
\end{description}


\begin{Schunk}
\begin{Sinput}
> NumAcc4 <- read.table(file=paste0(path,"NumAcc4.txt"))
> mean(NumAcc4$V1)
\end{Sinput}
\begin{Soutput}
[1] 10000000.199999999
\end{Soutput}
\begin{Sinput}
> sd(NumAcc4$V1)
\end{Sinput}
\begin{Soutput}
[1] 0.10000000055879354
\end{Soutput}
\begin{Sinput}
> acf(NumAcc4$V1, plot=F, lag.max=1)
\end{Sinput}
\begin{Soutput}
Autocorrelations of series 'NumAcc4$V1', by lag

     0      1 
 1.000 -0.999 
\end{Soutput}
\end{Schunk}

\section{NumAcc3 Data Set}
Using the NumAcc3.dat \url{http://www.itl.nist.gov/div898/strd/univ/numacc3.html} file. 

\subsection*{Data Set Description}

This generated/fabricated dataset consists of 1001 8-digit floating-point values: a single 1000000.2, followed by 500 pairings of 1000000.1 and 1000000.3. By construction, this data set has sample mean = 1000000.2 (exact); sample standard deviation = .1 (exact); and sample autocorrelation coef. = -0.999 (exact). The construction was carried out based on considerations described by Simon, Stephen D. and Lesage, James P. (1989): Assessing the Accuracy of ANOVA Caluclations in Statistical Software", Computational Statistics \& data Analysis, 8, pp. 325-332. 

\begin{description}
\item[] Data set properties
\item[Level of Difficulty] Average
\item[Variables] 1
\item[Observations] 101
\item[First Observation] 1000000.2
\end{description}

\begin{description}
\item[] Expected Results (as certified)
\item[Mean] 10000000.2 (exact)
\item[Standard Deviation]0.1 (exact)
\item[population lag-1 autocorrelation coefficient]-0.999       (exact) 
\end{description}


\begin{Schunk}
\begin{Sinput}
> NumAcc3 <- read.table(file=paste0(path,"NumAcc3.txt"))
> mean(NumAcc3$V1)
\end{Sinput}
\begin{Soutput}
[1] 1000000.2
\end{Soutput}
\begin{Sinput}
> sd(NumAcc3$V1)
\end{Sinput}
\begin{Soutput}
[1] 0.1000000000349246
\end{Soutput}
\begin{Sinput}
> acf(NumAcc3$V1, plot=F, lag.max=1)
\end{Sinput}
\begin{Soutput}
Autocorrelations of series 'NumAcc3$V1', by lag

     0      1 
 1.000 -0.999 
\end{Soutput}
\end{Schunk}

\section{NumAcc2 Data Set}
Using the NumAcc2.dat \url{http://www.itl.nist.gov/div898/strd/univ/numacc2.html} file.

\subsection*{Data Set Description}

This generated/fabricated dataset consists of 1001 2-digit floating-point values: a single 1.2, followed by 500 pairings of 1.1 and 1.3. By construction, this data set has sample mean = 1.2 (exact); sample standard deviation = 0.1 (exact); and sample autocorrelation coef. = -0.999 (exact). The construction was carried out based on considerations described by Simon, Stephen D. and Lesage, James P. (1989): Assessing the Accuracy of ANOVA Caluclations in Statistical Software", Computational Statistics \& data Analysis, 8, pp. 325-332.  

\begin{description}
\item[] Data set properties
\item[Level of Difficulty] Average
\item[Variables] 1
\item[Observations] 1001
\item[First Observation] 1.2
\end{description}

\begin{description}
\item[] Expected Results (as certified)
\item[Mean] 1.2 (exact) 
\item[Standard Deviation] 0.1 (exact)
\item[population lag-1 autocorrelation coefficient]-0.999       (exact) 
\end{description}




\begin{Schunk}
\begin{Sinput}
> NumAcc2 <- read.table(file=paste0(path,"NumAcc2.txt"))
> mean(NumAcc2$V1)
\end{Sinput}
\begin{Soutput}
[1] 1.2
\end{Soutput}
\begin{Sinput}
> sd(NumAcc2$V1)
\end{Sinput}
\begin{Soutput}
[1] 0.099999999999999978
\end{Soutput}
\begin{Sinput}
> acf(NumAcc2$V1, plot=F, lag.max=1)
\end{Sinput}
\begin{Soutput}
Autocorrelations of series 'NumAcc2$V1', by lag

     0      1 
 1.000 -0.999 
\end{Soutput}
\end{Schunk}


\section{NumAcc1 Data Set}
Using the NumAcc1.dat \url{http://www.itl.nist.gov/div898/strd/univ/numacc1.html} file.

\subsection*{Data Set Description}

This generated/fabricated dataset consists of three 8-digit integers differing only in the least significant digit. The data set is: 10000002, 10000001, and 10000003. By construction, this data set has sample mean = 10000002 (exact); sample standard deviation = 1 (exact); and sample autocorrelation coef. = -0.5 (exact). The construction was carried out based on considerations described by Simon, Stephen D. and Lesage, James P. (1989): Assessing the Accuracy of ANOVA Caluclations in Statistical Software", Computational Statistics \& data Analysis, 8, pp. 325-332

\begin{description}
\item[] Data set properties
\item[Level of Difficulty] Lower
\item[Variables] 1
\item[Observations] 3
\item[First Observation] 10000001
\end{description}

\begin{description}
\item[] Expected Results (as certified)
\item[Mean] 10000002 (exact)
\item[Standard Deviation] 1 (exact)
\item[population lag-1 autocorrelation coefficient]  -0.5           (exact)
\end{description}



\begin{Schunk}
\begin{Sinput}
> NumAcc1 <- read.table(file=paste0(path,"NumAcc1.txt"))
> mean(NumAcc1$V1)
\end{Sinput}
\begin{Soutput}
[1] 10000002
\end{Soutput}
\begin{Sinput}
> sd(NumAcc1$V1)
\end{Sinput}
\begin{Soutput}
[1] 1
\end{Soutput}
\begin{Sinput}
> acf(NumAcc1$V1, plot=F, lag.max=1)
\end{Sinput}
\begin{Soutput}
Autocorrelations of series 'NumAcc1$V1', by lag

   0    1 
 1.0 -0.5 
\end{Soutput}
\end{Schunk}

\section{Michelso Data Set}
Using Michelso.dat\url{http://www.itl.nist.gov/div898/strd/univ/michelso.html} file. 

\subsection*{Data Set Description}

This "real world" dataset is the result of the classic study conducted by Michelson on the speed of light in air in 1879. The response variable is speed of light (in millions of meters per second). The data was included as part of a larger study by Dorsey, Ernest N. (1944) on the velocity of light as reported in the Transactions of the American Philosophical Society.

\begin{description}
\item[] Data set properties
\item[Level of Difficulty] Lower
\item[Variables] 1
\item[Observations] 100
\item[First Observation] 299.85
\end{description}

\begin{description}
\item[] Expected Results (as certified)
\item[Mean] 299.852400000000
\item[Standard Deviation] 0.0790105478190518
\item[population lag-1 autocorrelation coefficient] 0.535199668621283
\end{description}



\begin{Schunk}
\begin{Sinput}
> Michelso <- read.table(file=paste0(path,"Michelso.txt"))
> mean(Michelso$V1)
\end{Sinput}
\begin{Soutput}
[1] 299.85239999999999
\end{Soutput}
\begin{Sinput}
> sd(Michelso$V1)
\end{Sinput}
\begin{Soutput}
[1] 0.079010547819050661
\end{Soutput}
\begin{Sinput}
> acf(Michelso$V1, plot=F, lag.max=1)
\end{Sinput}
\begin{Soutput}
Autocorrelations of series 'Michelso$V1', by lag

    0     1 
1.000 0.535 
\end{Soutput}
\end{Schunk}

\section{Mavro Data Set}
Using Mavro.dat\url{http://www.itl.nist.gov/div898/strd/univ/mavvo.html} file.

\subsection*{Data Set Description}
This "real world" dataset is the result of a study by Radu Mavrodineaunu, a chemist at the National Institute of Standards \& Technology (NIST). The purpose of the study was to determine a certified transmittance value that may be attached to the particular of filter under study. The 50 transmittance valuess were collected equi-spaced in time at a sampling rate of 10 observations per second.

\begin{description}
\item[] Data set properties
\item[Level of Difficulty] Lower
\item[Variables] 1
\item[Observations] 50
\item[First Observation] 2.00180
\end{description}

\begin{description}
\item[] Expected Results (as certified)
\item[mean] 2.00185600000000
\item[Standard Deviation] 0.000429123454003053
\item[population lag-1 autocorrelation coefficient]0.937989183438248
\end{description}



\begin{Schunk}
\begin{Sinput}
> Mavro <- read.table(file = paste0(path,"Mavro.txt"))
> mean(Mavro$V1)
\end{Sinput}
\begin{Soutput}
[1] 2.0018560000000001
\end{Soutput}
\begin{Sinput}
> sd(Mavro$V1)
\end{Sinput}
\begin{Soutput}
[1] 0.0004291234540030854
\end{Soutput}
\begin{Sinput}
> acf(Mavro$V1, plot=F, lag.max=1)
\end{Sinput}
\begin{Soutput}
Autocorrelations of series 'Mavro$V1', by lag

    0     1 
1.000 0.938 
\end{Soutput}
\end{Schunk}

\section{Lew Data Set}
Using Lew.dat \url{http://www.itl.nist.gov/div898/strd/univ/lew.html} file.

\subsection*{Data Set Description}

This "real world" dataset is the result of a study by H. S. Lew of the Structures Division of the Center for Building Technology at the National Institute of Standards \& Technology (NIST). The purpose of the study was to characterize the physical behavior of steel-concrete beams under periodic load. The response variable is deflection (from a rest point) of the steel-concrete beam. The 200 observations were collected equi-spaced in time. 

\begin{description}
\item[] Data set properties
\item[Level of Difficulty] Lower
\item[Variables] 1
\item[Observations] 200
\item[First observation] -213
\end{description}

\begin{description}
\item[] Expected Results (as certified)
\item[Mean] -177.435000000000
\item[Standard Deviation] 277.332168044316
\item[population lag-1 autocorrelation coefficient]-0.307304800605679
\end{description}
 


\begin{Schunk}
\begin{Sinput}
> Lew <- read.table(file=paste0(path,"Lew.txt"))
> mean(Lew$V1)
\end{Sinput}
\begin{Soutput}
[1] -177.435
\end{Soutput}
\begin{Sinput}
> sd(Lew$V1)
\end{Sinput}
\begin{Soutput}
[1] 277.33216804431612
\end{Soutput}
\begin{Sinput}
> acf(Lew$V1, plot=F, lag.max=1)
\end{Sinput}
\begin{Soutput}
Autocorrelations of series 'Lew$V1', by lag

     0      1 
 1.000 -0.307 
\end{Soutput}
\end{Schunk}

\section{Lottery}
Using Lottery.dat \url{http://www.itl.nist.gov/div898/strd/univ/lottery.html} file.

\subsection*{Data Set Description}


This dataset consists of 218 3-digit numbers (from 000 to 999) resulting from the state of Maryland's Pick-3 Lottery. The data was collected for the 32-week period September 3, 1989 to April 14, 1990. One 3-digit random number was drawn per day, 7 days per week for most weeks, but 6 or 5 days per week for other weeks. Interesting data-analytic questions involving the dataset are 1) are the lottery numbers uniformly distributed? and 2) is there serial correlation between lottery numbers?

\begin{description}
\item[] Data set properties
\item[Level of Difficulty] Lower
\item[Variables] 1
\item[Observations] 218
\item[First Observation] 162
\end{description}

\begin{description}
\item[] Expected Results (as certified)
\item[Mean] 518.958715596330
\item[Standard Deviation] 291.699727470969
\item[population lag-1 autocorrelation coefficient] -0.120948622967393
\end{description}
 
 

\begin{Schunk}
\begin{Sinput}
> Lottery <- read.table(file=paste0(path,"Lottery.txt"))
> mean(Lottery$V1)
\end{Sinput}
\begin{Soutput}
[1] 518.95871559633031
\end{Soutput}
\begin{Sinput}
> sd(Lottery$V1)
\end{Sinput}
\begin{Soutput}
[1] 291.69972747096909
\end{Soutput}
\begin{Sinput}
> acf(Lottery$V1, plot=F, lag.max=1)
\end{Sinput}
\begin{Soutput}
Autocorrelations of series 'Lottery$V1', by lag

     0      1 
 1.000 -0.121 
\end{Soutput}
\end{Schunk}

\section{Pi Digits Data Set}
Using PiDigits.dat \url{http://www.itl.nist.gov/div898/strd/univ/pidigits.html} file.
\subsection*{Data Set Description}

This dataset consists of the first 5000 digits of the mathemtatical constant pi (= 3.1415926535897932384...). These 5000 digits were reported in Mathematics of Computation, January 1962, page 76. Interesting number-theoretic questions involving pi digits are 1) are the digits uniformly distributed? and 2) is there serial correlation between successive digits?

\begin{description}
\item[] Data set properties
\item[Level of Difficulty] Lower
\item[Variables] 1
\item[Observations] 5000
\item[] All variables are all single digits.
\end{description}

\begin{description}
\item[] Expected Results (as certified)
\item[Mean] 4.53480000000000
\item[Standard Deviation] 2.86733906028871
\item[population lag-1 autocorrelation coefficient] -0.00355099287237972
\end{description}




\begin{Schunk}
\begin{Sinput}
> PiDigits <- read.table(file=paste0(path,"PiDigits.txt"))
> mean(PiDigits$V1)
\end{Sinput}
\begin{Soutput}
[1] 4.5347999999999997
\end{Soutput}
\begin{Sinput}
> sd(PiDigits$V1)
\end{Sinput}
\begin{Soutput}
[1] 2.8673390602887081
\end{Soutput}
\begin{Sinput}
> acf(PiDigits$V1, plot=F, lag.max=1)
\end{Sinput}
\begin{Soutput}
Autocorrelations of series 'PiDigits$V1', by lag

     0      1 
 1.000 -0.004 
\end{Soutput}
\end{Schunk}



\end{document}
